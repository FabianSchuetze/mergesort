\documentclass[xcolor=table]{beamer}
\usepackage{lmodern}
%\usepackage[normalem]{ulem}
\usepackage{ marvosym }
\usepackage[export]{adjustbox}
\usepackage{mathtools,calc}
\newcommand\Fontvi{\fontsize{22}{23.2}\selectfont}
\newcommand\strikeout[2][]{%
 \begin{tabular}[b]{@{}c@{}}
    \makebox(0,0)[cb]{\textcolor{blue}{#1}} \\[-0.2\normalbaselineskip]
     \rlap{\color{red}\rule[0.5ex]{\widthof{#2}}{0.5pt}}#2
 \end{tabular}}

\newcommand<>\mathalt[2]{%
  \alt#3{\mathmakebox[\widthof{$#2$}]{#1}}{#2}%
}
\usepackage{colortbl}
\usepackage{booktabs}
%\usepackage{xcolor}
%\usepackage[usenames, dvipsnames]{color}
\definecolor{red}{rgb}{0.894, 0.101, 0.109} 
\definecolor{Gray}{gray}{0.85}
\newcolumntype{a}{>{\columncolor{Gray}}c}
\definecolor{green}{rgb}{0.1054, 0.6171, 0.4648}
\definecolor{orange}{rgb}{0.8476, 0.3711, 0.0078}
\definecolor{violet}{rgb}{0.9023, 0.1602, 0.5391}
%\usepackage[doi=false, backref=true, url=false, isbn=false, backend=biber,
						%citestyle=authoryear]{biblatex}
\usepackage{tabu}
\newcommand{\blue}{\textcolor{blue}}
\newcommand{\red}{\textcolor{red}}
\newcommand{\green}{\textcolor{green}}
\newcommand{\organge}{\textcolor{orange}}
\newcommand{\violet}{\textcolor{violet}}
% Just for demo

\AtBeginSection[]{
  \begin{frame}
  \vfill
  \centering
  \begin{beamercolorbox}[sep=8pt,center,shadow=true,rounded=true]{title}
    \usebeamerfont{title}\insertsectionhead\par%
  \end{beamercolorbox}
  \vfill
  \end{frame}
}
\usepackage{mathtools}
\newtheorem{prop}{Proposition}
\newtheorem{hypothesis}{Hypothesis}
\usepackage{amsmath}
\usepackage{multirow}
\usepackage{makecell} % to make linebreaks in table
\renewcommand\theadalign{bc}
\renewcommand\theadfont{\bfseries}
\renewcommand\theadgape{\Gape[4pt]}
\renewcommand\cellgape{\Gape[4pt]}
\usepackage{hhline}
\usepackage{geometry}
\usepackage{booktabs}
\usepackage[]{hyperref}%
\hypersetup{colorlinks, linkcolor={blue}, citecolor={blue}, urlcolor={red}}
\usepackage{graphicx}

\DeclareMathOperator*{\argmax}{\arg\!\max}
\DeclareMathOperator{\E}{\mathbb{E}}
%\addbibresource[]{library.bib}
\usepackage[utf8]{inputenc}
\usepackage[T1]{fontenc}
\setbeamertemplate{caption}{\raggedright\insertcaption\par}
\usetheme{default}
\usepackage[flushleft]{threeparttable}
\usepackage{booktabs,caption,fixltx2e}
\usepackage{ragged2e}
\usepackage{appendixnumberbeamer}
\makeatletter
\usefonttheme{professionalfonts}
\usefonttheme{serif}
\setbeamertemplate{navigation symbols}{}%remove navigation symbols
\setbeamertemplate{footline}
{
  \leavevmode%
  \hbox{%
  \begin{beamercolorbox}[wd=.333333\paperwidth,ht=2.25ex,dp=1ex,center]{author in head/foot}%
    \usebeamerfont{author in head/foot}\insertsection\
  \end{beamercolorbox}%
  \begin{beamercolorbox}[wd=.333333\paperwidth,ht=2.25ex,dp=1ex,center]{title in head/foot}%
    \usebeamerfont{title in head/foot}\insertsubsection\
  \end{beamercolorbox}%
  \begin{beamercolorbox}[wd=.333333\paperwidth,ht=2.25ex,dp=1ex,right]{date in head/foot}%
    %\usebeamerfont{date in head/foot}\insertshortdate{}\hspace*{2em} % hide %date
    %\insertframenumber{} / \inserttotalframenumber\hspace*{2ex} 
  \end{beamercolorbox}}%
  \vskip0pt%
}
\makeatother
%\setbeamertemplate{footline}[frame number]{}


\usepackage{listings}
\lstset{
    tabsize=4,
    showstringspaces=false,
    numbers=left,
    commentstyle=\color{green},
    keywordstyle=\color{blue},
    stringstyle=\color{red}
}


 %----------------------------------------------------------------------------------------
% %	TITLE PAGE
% %----------------------------------------------------------------------------------------
\renewcommand*{\thefootnote}{\fnsymbol{footnote}}
\title[]{Ideas for Low Celluclast Performance}
\date{13.11.2019} % Date, can be changed to a custom date
%\author{Fabian Schuetze, EUI}
\addtobeamertemplate{block begin}{}{\justifying}  %new code
\setbeamertemplate{itemize items}[circle]

\begin{document}

\begin{frame}
\titlepage
\end{frame}

\section{Basic Architecture}

\begin{frame}
\begin{itemize}
    \item  Accesses different memory than CPU
    \item Starting thousands of threads at low costs

\end{itemize}
\end{frame}

\begin{frame}[fragile]
\frametitle{Memory Management}
\begin{lstlisting}[language=c++]
typedef int dtype;
class Storage {
   public:
    explicit Storage(const std::vector<int>&);

   private:
    std::vector<int> _data;
    dtype* _cpu_pointer;
    dtype* _gpu_pointer;
    void initialize_gpu_memory();
};
\end{lstlisting}
\begin{itemize}
    \item Memory pool, takes ownership
    \item Initializes the gpu memory as copy
    \item two differetn pointer, to cpu/gpu locations
\end{itemize}
\end{frame}

\begin{frame}[fragile]
\begin{lstlisting}[language=c++]
typedef int dtype;
class Storage {
   public:
    explicit Storage(const std::vector<int>&);
    const dtype* cpu_pointer_const();
    const dtype* gpu_pointer_const();
    dtype* cpu_pointer();
    dtype* gpu_pointer();

   private:
    std::vector<int> _data;
    dtype* _cpu_pointer;
    dtype* _gpu_pointer;
    void initialize_gpu_memory();
    std::string recent_head;
    void sync_to_cpu();
    void sync_to_gpu();
};
\end{lstlisting}
\begin{itemize}
    \item accesses pointers
    \item $_recent_head$ specifies on which device memory changed last
    \item if non-const pointer, expects modificaion, and, change head
    \item access pointer out-of-sync, sync to location
\end{itemize}
\end{frame}

\section{Merge}
\begin{frame}
    Write about CPU merge
\end{frame}

\begin{frame}
    \frametitle{How to spwan to many threads?}
Paper that does that, show as example, the cutting approach
Naiv approach : 2 Threads Cut both a and b into half, 
\begin{align*}
    A &= \begin{matrix} 0 0 0 0 \end{matrix}
    B &= \begin{matrix} 1 1 1 1 \end{matrix}
    C &= \begin{matrix} ? ? ? ? ? ? ? ? \end{matrix}
\end{align*}
\begin{align*}
    A &= \begin{matrix} \underbrace{0 0}_{\text{Thread 1}} | 
                        \underbrace{0 0}_{\text{Thread 2}} \end{matrix}
    B &= \begin{matrix} \underbrace{1 1}_{\text{Thread 1}} | 
                        \underbrace{1 1}_{\text{Thread 2}} \end{matrix}
    C &= \begin{matrix} \underbrace{? ? ? ?}_{\text{Thread 1}} | 
    \underbrace{? ? ? ?}_{\text{Thread 2}} \end{matrix}
\end{align*}
    \begin{align*}
    C &= \begin{matrix} \underbrace{0 0 1 1}_{\text{Thread 1}} | 
    \underbrace{0 0 1 1}_{\text{Thread 2}} \end{matrix}
\end{align*}
\end{frame}

\begin{frame}
    \frametitle{How to allocated work?}
    Here are the mergepath pictures
\end{frame}

\begin{frame}[fragile]
    \frametitle{Comutation Procedure}
\begin{lstlisting}[language=c++]
__global__ void paralleMerge3(int* a, int sz_a, int* b, int sz_b, int* c,
                              int length) {
    int diag = threadIdx.x * length;
    int a_start = mergepath(a, sz_a, b, sz_b, diag);
    int b_start = diag - a_start;
    merge2(a, a_start, sz_a, b, b_start, sz_b, c, diag, length);
}
\end{lstlisting}
\begin{itemize}
    \item Each tread works on one part
    \item Thread calculates the value $A_lower$ for itself
    \item Calcultes the also the $B_lower$ (Why does that work again?)
    \item merges the two arrays
\end{itemize}
\end{frame}


\begin{frame}
\frametitle{Problem: Slow as a Snail}
show the growth rates vs std::mergesort
\end{frame}

\begin{frame}
\frametitle{Reason: So much global meory access}
50 Percent of the global memory traffic is caused by 3 Percent of the values\\
Corrobation: Cuda performance tool
\end{frame}

\section{Memory Hirachy of CUDA}
\begin{frame}
\frametitle{The different memories and their sizes}
show the plot of the different memories and their relative size on my card
\end{frame}

\section{Merging with local memory}
\begin{frame}
\frametitle{Describe the shared memory}
show the plot of the different memories and their relative size on my card
\end{frame}

\begin{frame}
\frametitle{Show the results}
show the plot of the different memories and their relative size on my card
\end{frame}
\end{document}
